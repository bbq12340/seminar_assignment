\documentclass[10pt]{article}
\usepackage{graphicx} % Required for inserting images

\title{HW1_2023-23441}
\author{장재혁 / 학생 / 통계학과 ­}
\date{March 2023}

\begin{document}
\section*{Measures on the real line}
\textbf{Lemma 1.1.10} Suppose $\mu$ defined on semialgebra $S$ have $\mu(\emptyset)=0$ satisfy finite additivity of $\mu$ on $\mathcal{S}$ i.e. if $S \in \mathcal{S}$. If $\bar{\mu}$ is the unique extension of $\mu$ on $\bar{\mathcal{S}}$, the algebra generated by $\mathcal{S}$, then $\bar{\mu}(A_i) \leq \Sigma_i\bar{\mu}(B_i)$ where $A, B_i \in \bar{\mathcal{S}}$ s.t. $A \subset \cup_{i=1}^n B_i$.\\
\textit{Proof of Lemma 1.1.10} Suppose if $S_i \in \bar{\mathcal{S}}$ s.t. $A=+_iS_i$, then for each $S_i$, $\exists S_{i,j}\in\mathcal{S}: +_jS_{i,j}=S_i$. So,
$$\bar{\mu}(A)=\Sigma_{i,j}\mu(S_{i,j})=\Sigma_i\bar{\mu}(S_i)$$
Set $F_1=B_1$ and $F_n = B_n - \cup_1^{n-1}B_i$ so that 
$$\cup_i B_i = F_1 + \cdots F_n$$
$$A = A\cap(\cup_i B_i) = (A\cap F_1) + \cdots + (A\cap F_n)$$
So using the finite additivity and the fact above,
$$\bar{\mu}(A)=\Sigma_{k=1}^n\bar{\mu}(A\cap F_k) \leq \Sigma_{k=1}^n\bar{\mu}(F_k)=\bar{\mu}(\cup_i B_i)$$
\textbf{Theorem 1.1.9} Let $\mathcal{S}$ be a semialgebra and let $\mu$ defined on $\mathcal{S}$ have $\mu(\emptyset)=0$. Suppose $\mu$ satistfy (i) Finite Additivity, (ii) Countable Subadditivity i.e. if $S_i,S\in\mathcal{S}$ with $S=+_{i\geq1}S_i$, then $\mu(S)\leq\Sigma_{i\geq1}\mu(S_i)$. Then $\mu$ has a unique extension $\bar{\mu}$ that is a measure on $\bar{\mathcal{S}}$, the algebra generated by $\mathcal{S}$. If $\bar{\mu}$ is a sigma finite, then there is a unique extension $\nu$ that is a measure on $\sigma(\mathcal{S})$.\\
The proof of Theorem 1.1.9 is a long road which is given in Section A.1 in \textit{Durrett, Proability Theory and Examples, 5th Ed.}\\
\textbf{Theorem 1.1.4} Associated with each Stieltjes measure function $F$ there is a unique measure $\mu$ on $(R,\mathcal{R})$ with $\mu((a,b])=F(b)-F(a)$
\begin{equation}
    \mu((a,b])=F(b)-F(a)
\end{equation}
When $F(x)=x$, the resulting measure is called \textbf{Lebesgue measure}.\\
\textit{Proof of Theorem 1.1.4} Let $\mathcal{S}$ be the semialgebra of half-open intervals $(a,b]$ with $-\infty \leq a < b \leq \infty$. To define $\mu$ on $\mathcal{S}$, we begin by observing that 
\begin{center}
    $F(\infty)=lim_{x\uparrow \infty}F(x)$ and $F(-\infty)=lim_{x\downarrow -\infty}F(x)$ exist
\end{center}
and $\mu((a,b])=F(b)-F(a)$ makes sense for all $-\infty \leq a < b \leq \infty$ since $F(\infty) > -\infty$ and $F(-\infty) < \infty$.\\
If $(a,b] = +_{i=1}^n(a_i,b_i]$, then after relabeling the intervals we must have $a_1=a, b_n=b$ and $a_i=b+{i-1}$ for $2 \leq i \leq n$ so that it satisfy the finite additivity of $\mu$ on $\mathcal{S}$ i.e. if $S \in \mathcal{S}$, is a finite disjoint union of sets $S_i \in \mathcal{S}$, then $\mu(S)=\Sigma_i \mu(S_i)$. Moreover, if $\mu$ satisfy the countable subadditivity, then by Theorem 1.1.9, there is a unique measure on $\sigma(S)$. To check the countable subadditivity, suppose first that $-\infty < a < b < \infty$ and $(a,b] \subset \cup_{i\geq1}(a_i,b_i]$ where (without loss of generality) $\infty < a_i < b_i <\infty$. Pick $\delta > 0$ so that $F(a+\delta) < F(a) + \epsilon$ and pick $\eta_i$ so that
$$F(b_i + \eta_i) < F(b_i) + \epsilon2^{-i}$$
The open intervals $(a_i, b_i + \eta_i)$ cover $[a + \delta, b]$, so there is a finite subcover $(\alpha_j, \beta_j), 1\leq j \leq J$. Since $(a+\delta, b] \subset \cup_{j=1}^J(\alpha_j, \beta_j]$, Lemma 1.1.10 implies 
$$F(b)-F(a+\delta) \leq \Sigma_{j=1}^J F(\beta_j) - F(\alpha_j) \leq \Sigma_{i=1}^\infty (F(b_i+\eta_j) - F(a_i))$$
So, by the choice of $\delta$ and $\eta_i$,
$$F(b)-F(a) \leq 2\epsilon + \Sigma_{i=1}^\infty(F(b_i) - F(a_i))$$
and since $\epsilon$ is arbitrary, we have proved the result in the case $-\infty < a <b < \infty$. To remove the last restriction, observe that if $(a,b] \subset \cup_i(a_i, b_i]$ and $(A, B] \subset (a,b]$ has $-\infty < A < B < \infty$, then we have 
$$F(B) - F(A) \leq \Sigma_{i=1}^\infty(F(b_i) - F(a_i))$$
Since the last result holds for any finite $(A,B] \subset (a,b]$, the desired result follows.
\end{document}
